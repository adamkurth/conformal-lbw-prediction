\newpage 
\chapter{Introduction}
\label{chap:introduction}

\section{Background \& Problem}
%Importance of low birth weight (LBW) as a public health issue
%Impact on infant mortality, developmental health, and long-term outcomes
%Need for statistical modeling to understand determinants of LBW

Low birth weight (LBW), defined as a birth weight less than 2.5 kilograms \parencite{lbw_def, kramer1987}, is a significant public health indicator. Infants born with LBW face substantially higher risk for neonatal mortality, developmental delays, and chronic health problems such as respiratory and neurological impairments \parencite{finch2003}. These adverse outcomes arise from a complex interplay of genetic, biological, environmental, and socioeconomic factors. Understanding and identifying determinants of LBW is therefore critical for developing informing targeted preventative measures and improving neonatal outcomes. 

Decades of research confirm that LBW is a multifactorial issue. An early landmark meta-analysis by \textcite{kramer1987} reviewed 895 studies from 1974-1984, and identified 43 causal determinants. Kramer concluded that maternal anthropometry (height and pre-pregnancy weight), inadequate gestational weight gain, cigarette smoking, malaria infection, and a history of adverse pregnancy outcomes exert \emph{independent} effects on intrauterine growth restriction (IUGR), while few factors influence gestational duration \parencite{kramer1987}. The highly-interrelated nature of these risk factors led to confounding, yielding an impediment for modeling birth-weight outcomes by their interactive, not additive, effects. For example, inadequate pregnancy weight gain might depend on whether she smokes or has health conditions. Classical regression models such as logistic regression, typically assume additive effects and thus miss such interactions. As a result, traditional models often struggle to disentangle which combinations of maternal-infant characteristics \emph{truly} signify an at-risk pregnancy.

Subsequent work in epidemiology repeatedly show that these risk factors are not found in isolation. In \citeyear{KITSANTAS2006275}, \textcite{KITSANTAS2006275} use Classification and Regression Trees (CART) developed by \textcite{breiman1984classification} to identify high risk profiles of LBW on a large dataset of Florida birth records, uncovering important context-specific combinations that influence LBW risk. For instance, mothers who smoked \emph{and} had inadequate weight-gain during pregnancy, had sharply elevated LBW risk. This study highlights the strengths of interpretability using CART, but still lacks predictive power over logistic regression by relying entirely on empirical observations. Moreover, reducing the problem to a binary LBW indicator variable discards vital information about how far a newborn falls below the LBW threshold. Regardless of differences, a baby just above 2.5 kg is treated the same as a much heavier baby, and all LBW cases are treated alike. This binary cutoff masks important differences in the birth-weight distribution.

Recent research has moved beyond classification toward estimating the full birth weight distribution conditional on covariates. Bayesian nonparametric mixture models allow the birth weight density to vary flexibly across subpopulations defined by maternal factors, without strong parametric assumptions \parencite{dunson2008}. Other approaches use copula-based or density regression techniques to jointly model birth weight with related outcomes such as gestational age \parencite{rathjens2023}. These methods can capture detailed distributional effects of predictors, such as how covariates influence the entire left tail of the birth weight distribution. However, a drawback in these advanced models is their complexity and lack of interpretability for users. In contrast, practitioners particularly in public health often prefer models that yield clear, simple decision rules for identifying high-risk subgroups. 

Several demographic and socioeconomic factors are well-known to influence LBW risk. Younger and older mothers are associated with higher incidence of LBW \parencite{age_differences_lbw}. Lower educational attainment is associated with limited healthcare access \parencite{finch2003,jain2024}, and environmental exposures, including tobacco use, substance abuse, and air pollution, further elevate risk by interfering with fetal growth and development \parencite{standford_med_lbw, lu2020combined}. Inadequate prenatal care is another important factor of LBW outcomes \parencite{prenatal_lbw}. Crucially, LBW incidence also varies sharply by racial and economic contexts. In the United States, the LBW incidence rate for Black infants is double that of white newborns, comparing 14.7\% to 7.1\%  \parencite{marchofdimes2024}. These patterns underscore the multifactorial nature of effects that influence LBW outcomes and the need to account for diverse influences in any predictive model. 

Taken together, these considerations highlight a central challenge in birth weight modeling: existing methods trade flexibility for interpretability. Approaches using decision trees alone provide transparent subgroup rules while ignoring the full birth-weight distributions, while advanced Bayesian density models capture distributional details but lack intuitive clarity. This work aims to bridge the gap by developing a Bayesian tree-based framework that stratifies the population into interpretable risk subpopulations while modeling full birth-weight distributions for predicting LBW outcomes.
