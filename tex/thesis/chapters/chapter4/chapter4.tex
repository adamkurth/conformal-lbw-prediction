\chapter{Conclusion \& Future Work}
\label{chap:conclusion}

This study introduces a Bayesian tree-based methodology to investigating determinants of LBW using a nationally representative dataset. By the integration of the DM likelihood into the CART framework, the model addresses both data scarcity in rare outcome classes, with addressing the need for a more flexible and interpretable modeling structure. Using historic data, the quantiles inform the priors binning procedure, creating the necessary birth-weight categories. The quantile-based categories enable the the informed prior to represent subtle gradients in the LBW-region, and enabling detection of distributional shifts given a set of predictors. Additionally, the proposed bootstrap methodology handles the consolidated counts data, while the results yield stable and reliable estimates across the ensemble. 

A consistent theme in this project is that maternal race, marital status, and smoking status were dominant indicators of LBW risk. Furthermore, the restricted LBW-only model shifted the focus from socioeconomic and demographic predictors to biological and behavior-based variables. Such variables include maternal age, infant gender, and prenatal care. Note that these findings align with epidemiological literature, demonstrating the utility of our proposed modeling framework to extract and interpret clinically relevant rules from high-dimensional data. 

However, the present analysis is bounded by a constrained set of binary predictors and reductionist encoding of sociodemographic and behavioral traits. In further analysis, clinically relevant information could be utilized instead of discarded by encoding predictors into binary. Further, the race predictor currently represents a coarse proxy of demographic and socioeconomic conditions in further studies should refine this important predictor to capture more social and cultural dimensions. The most natural next step is enhancing the data with a richer predictor set of maternal-infant health and environmental indicators. Promising health related variables include: history of hypertension \parencite{hypertension_lbw}, diabetes \parencite{diabetes_lbw}, BMI \parencite{bmi_lbw} , mental health \parencite{mental_lbw}, and prior pregnancy complications \parencite{prior_lbw}. Including these variables could enhance the model's ability to further classify at-risk subgroups, since they are all known to affect fetal growth. Incorporating environmental and contextual variables can expand the model's scope and abilities. Structural determinant such as air quality metrics, neighborhood crime rates, housing conditions, food accessibility, and proximity to prenatal services may interact with biological and behavioral risks in meaningful ways. Their inclusion would support a more holistic understanding of LBW outcomes. Also, the temporal trends of any of these variables is worth while to investigate due to the consistent annual reporting of the natality dataset.

Moreover, this modeling framework has the potential to be enhanced as a practical tool for clinical triage or public health screening. Future work should focus on adapting the modeling framework into a practitioner-friendly risk calculator suitable for intake assessments or integration into electronic health records. This would significantly enhance accessibility for health practitioners and support early identification of at-risk pregnancies. 

This work contributes a flexible and interpretable modeling approach for modeling LBW determinants and lays the foundation for future interdisciplinary research that intersects statistical modeling, clinical practice, and public health policy. Expanding the set of predictors and translating the model into operational tools would be critical steps toward leveraging these insights into actionable health interventions.
