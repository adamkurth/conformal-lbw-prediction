\begin{center}
    {\large \textbf{ABSTRACT}}
\end{center}
Low birth weight (LBW) remains a critical public-health indicator, linked strongly with higher neonatal mortality, developmental delays, and lifelong chronic diseases. Using the 2021 U.S. Natality dataset (> 3 million births), this thesis develops a Bayesian, tree-based, nonparametric framework that models the full birth weight distribution and quantifies LBW risk.

The raw dataset is condensed into 128 mutually exclusive classes defined by seven dichotomous maternal-infant predictors and 11 birth weight categories, comprised of 10\% LBW quantile categories plus one aggregated normal weight category for added LBW granularity. Classification and Regression Trees (CART) are grown using the marginal Dirichlet-Multinomial likelihood as the splitting criterion. This criterion is equipped to handle sparse observations, with the Dirichlet hyperparameters informed by previous quantiles from the 2020 dataset to avoid "double dipping".

Employing a two-tier parametric bootstrap resampling technique, a 10,000 tree ensemble is grown yielding highly stable prediction estimates. Maternal race, smoking status, and marital status consistently drive the initial LBW risk stratification, identifying Black, smoking, unmarried mothers among the highest-risk subgroups. When the analysis is restricted to LBW births only, infant sex and maternal age supersede smoking and marital status as key discriminators, revealing finer biological gradients of risk. Ensemble predictions are well calibrated, and 95\% bootstrap confidence intervals achieve nominal coverage.

The resulting framework combines the interpretability of decision trees with Bayesian uncertainty quantification, delivering actionable, clinically relevant insights for targeting maternal-health interventions among the most vulnerable subpopulations.