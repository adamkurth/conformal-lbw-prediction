\section{Previous Work on Birth-Weight Modeling}
\label{sec:ch2-previous-work}

As previously mentioned, \textcite{kramer1987}'s meta-analysis identified 43 LBW determinants then categorized them into genetic, nutritional, psychosocial, etc. and assessed their effects on birth weight and prematurity. Separated based on income status, for high-income mothers, smoking status, poor maternal nutrition or low pre-pregnancy weight were the strongest LBW determinants whereas in low-income settings, maternal race origin, undernutrition, short stature, and malaria exposure were found to be the most important predictors \parencite{kramer1987}. While for preterm births, smoking status and low pre-pregnancy weight are strong indicators \parencite{kramer1987}.  \textcite{kramer1987} concluded by stating that many potential contributors remain under studied, naming maternal work, prenatal care, and previous infections as some examples. This comprehensive seminal work highlights the complex and multifactorial nature of LBW, leaving open questions about interactions of factors and distributional outcomes, motivating a more flexible modeling procedure.

The application of CART by \textcite{KITSANTAS2006275} to 181,690 singleton births from Florida, led the identification of high-risk LBW mothers. Known risk factors of smoking status, gestation weight-gain, parity, etc. were used to grow separate decision trees by geographic region, and compared against logistic regression \parencite{KITSANTAS2006275}. The CART model revealed high-risk profiles for White and Hispanic mothers with low pregnancy weight gain, parity, and marital status defined high-risk stratification among non-smokers \parencite{KITSANTAS2006275}. For instance, smoking mothers that gain less than 20 lbs are at significantly higher risk than mothers of larger weight-gain pregnancies, and Black mothers form a high-risk subpopulation in some regions regardless of other factors \parencite{KITSANTAS2006275}. However, predictive accuracy was marginally better than logistic regression \parencite{KITSANTAS2006275}, the recursive partitioning procedure conducted by CART uncovered some of the complex factor interaction in the LBW data. This study shows how the order of factors could be useful in disentangling strong interaction effects, suggesting room for improved or alternative methods. 

\textcite{dunson2008} (\citeyear{dunson2008}) used Bayesian semiparametric methods to link maternal pregnancy weight gain to birth weight distributions. Using a Dirichlet-process mixture, they flexibly defined clusters of women by their weight-gain trajectories and jointly modeled birth weight densities across clusters \parencite{dunson2008}. This approach allowed the \emph{entire} birth weight distribution to vary with weight-gain patterns, including distribution tails, while also capturing heterogeneity of how pregnancy factors influence birth weight \parencite{dunson2008}. Dunson et. al. demonstrated that modeling the full distribution in perinatal data is insightful — beyond mean estimates. However, advanced Bayesian models, latent clustering, and complex MCMC procedures lack interpretability and are computationally intensive, highlighting the need for model simplicity while retaining flexibility.

More recently, \textcite{rathjens2023} in \citeyear{rathjens2023} proposed a Bayesian distribution regression approach using copulas to jointly model birth weight and gestational age. Marginal distributions are assumed to follow a Gaussian for birth-weight outcomes, and Dagum distribution for skewed gestational age, and the copula linked the two cumulative distribution functions (CDF) as functions of covariates \parencite{rathjens2023}. The results of this study show non-linear effects of gestational age on weight and tail-dependent associations were captured by a Clayton copula \parencite{rathjens2023}. The focus of bivariate outcomes here shows how distribution modeling can extend traditional regression approaches. Beyond complex copula models, Bayesian methods enrich perinatal risk modeling. 

In \citeyear{jain2024}, \textcite{jain2024} proposed a scalable Bayesian density estimation method for nationally collected birth records. Inspired by kernel density methods, a Gaussian mixture is employed to model conditional distributions of birth weights given various predictors. Through advanced MCMC and targeted subsampling techniques, the model was able to capture complex patterns and estimate birth-weight densities at scale. Jain's work estimates the full distribution by density regression, but underscores the computational and interpretational challenges.

\section{Current Approach \& Contributions}
\label{sec:ch2-current-approach}

In this thesis, we adopt CART and Bayesian nonparametric methods to approximate birth-weight distributions. CART is a nonparametric algorithm proposed by \textcite{breiman1984classification} and implemented in R by Therneau et al. \parencite{intro_to_rpart}, called \textit{Recursive Partitioning and Regression Trees}, or \texttt{rpart}. The algorithm works in two stages: tree construction and tree pruning. 
 
First, the tree is constructed. Given the data, \texttt{rpart} recursively partitions it into binary splits on the given predictor variables, creating nodes at each split. Though the splits need not be binary, this provides a clear and interpretable tree. CART employs a greedy approach to building decision trees \parencite{cart_greedy}, where its goal is to maximize homogeneity, or equivalently minimize heterogeneity in the data. At each node, CART evaluates all possible splits on candidate predictors and chooses the one that best "explains" the data by minimizing the node impurity, resulting in two child nodes with more homogeneous subgroups \parencite{intro_to_rpart}. This process is applied recursively to each child node then growing a larger tree until the tree's max depth is reached or no further improvement is found \parencite{intro_to_rpart}.

Once fully grown, the tree typically overfits to the data, yielding large errors for small fluctuations. To address this, cross-validation is used to estimate prediction error for a sequence of pruned trees \parencite{intro_to_rpart}. The tree is then "trimmed" back to the best cross-validation performance \parencite{intro_to_rpart}, yielding the final tree that balances complexity and accuracy. For each terminal node (or "leaf") in the final tree, a sequence of if-then conditions categorize birth-weight outcomes based on maternal covariates.

Interpretability is preserved by using CART to automatically uncover high- and low-risk groups for subpopulations of specific maternal and infant characteristics, much akin to \textcite{KITSANTAS2006275}. Additionally, in line with \textcite{dunson2008} and \textcite{jain2024}, this tree-based method imposes no strict distributional assumptions on the birth-weight responses allowing for nonlinear interactions and heterogeneous effects to be captured naturally by CART. Our preprocessing procedure results in count data of various birth-weight categories, motivating the use of the \emph{marginal Dirichlet-Multinomial (DM) likelihood} as the Bayesian "evidence" and splitting criterion. The DM likelihood is chosen by producing posterior predictive distributions and interval estimation at each leaf whereas the Gini index measures only impurity. The impurity of a terminal node, is entirely dependent on the sample size by relying on empirical proportions \parencite{stackexchangeGiniDecrease}. Additionally, the Gini index is known to suffer with data sparsity \parencite{ekamperiDecisionTrees}, and compared to normal birth-weight (NBW) observations, we expect a large discrepancy between the total number of observed LBW and NBW counts.

Birth-weight count observations can be safely assumed to follow a \emph{multinomial} distribution, and the Dirichlet prior smooths categories not observed. For use in CART, a split with high DM likelihood translates as added improvement from parent to child nodes, reducing heterogeneity. The DM likelihood will be derived formally in Section~\ref{sec:ch2-likelihood} as a favorable splitting criterion for our application.

\section{Overview of the Birth Weight Dataset}
\label{sec:ch2-introduction}

The primary dataset for this analysis is the 2021 Vitality Statistics Natality Birth Data \parencite{nber_birth_data}. Collected by the National Center for Health Statistics (NCHS), this dataset contains a detailed record of birth outcomes and various maternal characteristics as part of the Vital Statistics Cooperative Program \parencite{jain2024, nber_birth_data}. Standing as one of the most comprehensive datasets with over 3 million birth-weight records for maternal and infant health in the United States, collected annually across all states and District of Columbia since 1972 \parencite{nber_birth_data}.

For this analysis, variables in the 2021 data are broadly categorized into three domains: demographic, health, and geographic. Demographic features include date of birth, parental age and education, marital status, birth order, sex, and geographic location. Health features cover birth weight, gestational age, prenatal care adequacy, delivery attendants, and Apgar scores, while geographic indicators include state, county, and metropolitan  status \parencite{nber_birth_data}. Note that Apgar scores are examinations based on newborn vitals five minutes following delivery, observing how newborns handle being outside the mother's womb \parencite{apgar_score}.

\section{Data Preprocessing \& Feature Engineering}
\label{sec:ch2-preprocessing}

The preprocessing procedure transforms the high-dimensional 2021 dataset into a workable and condensed dataset for computational efficiency, while preserving key information about predictors. Preprocessing involved (1) encoding all categorical and continuous variables into unique dichotomous predictors, (2) dimension reduction from 3 million rows to 128 unique predictor combinations, and (3) creating a consolidated counts dataset, primarily expanding the LBW region by creating birth-weight categories based on quantile cut points. From the dataset, seven key predictor variables and birth-weight outcomes (in kg) are retained for modeling. 

\subsection{Binary Feature Encoding}
\label{sec:ch2-feature-encoding}

To enhance interpretability and computational efficiency, only seven predictors are selected based on clinical relevance, strong generalizability,  and prior research support. The encoding procedure was inherited from \textcite{jain2024}, and these predictors serve as an example of a small, yet representative set of predictors. Note that the encoding of \texttt{mrace15} is suggested by \textcite{jain2024} and \textcite{marchofdimes2024} as the primary dichotomy, \emph{though this choice is entirely arbitrary}.  According to \citeyear{census_usa} \textcite{census_usa}, the national population is roughly 75.3\% White and 13.7\% Black, which provides demographic context for this binary split. All information of each feature representation and meaning is conveyed in the table below.

\begin{table}[ht]
\centering         % ensures horizontal centering
\small             % or \footnotesize
\caption{Binary predictor definitions used in this study}
\begin{tabular}{@{}llll@{}}
\toprule
\textbf{Label} & \textbf{Natality field} & \textbf{Value = 1} & \textbf{Value = 0} \\ \midrule
Boy          & \texttt{sex}      & Infant is male (“M”)                     & Infant is female (“F”) \\
Married      & \texttt{dmar}     & Mother is married                        & Mother not married     \\
Black        & \texttt{mrace15}  & Black / African American                 & Any other race         \\
Over33       & \texttt{mager}    & Maternal age $>$ 33 yr                   & Maternal age $\le$ 33 yr \\
HighSchool   & \texttt{medu}     & High-school education completed          & Otherwise              \\
FullPrenatal & \texttt{prenatal} & Adequate prenatal care                   & Inadequate / none      \\
Smoker       & \texttt{cig\_0}   & Any prenatal smoking                     & No smoking             \\ \bottomrule
\end{tabular}
\end{table}



\subsection{Dimension Reduction}
\label{sec:ch2-dimension-reduction}

After the first step in preprocessing, the data is encoded as dichotomous indicator variables and one response column of total recorded birth weight outcomes for the 3 million records. There are \(2^7 = 128\) possible combinations for the predictors, each representing a unique \emph{class} of maternal and infant characteristics. Aggregating observations by class greatly reduces the computational load while preserving interpretability and necessary information of features, enabling discernment of risk factors with minimal computational burden.

\subsection{Converting Birth Weight into Count Variables} 
\label{sec:ch2-count-variables}

The final step, transforms the dataset from 3.6 million by 237 to 128 by 11. We define a sequence of decline-quantile cut points based on 10\% quantile increments, to segment the LBW region (from 0-to-2.5 kg) creating 10 total LBW categories. By using the previous year's identical dataset from 2020 in segmenting these categories, we eliminate problems with "double-dipping" and bias in later estimates. This dimension reduction drastically consolidates the dataset while providing a straightforward way to retrieve the number of observations within a given class and birth-weight category. The specific cut-point values and their prior assignment are shown in Table~\ref{tab:birthweight_quantiles}.

% insert table of quantile cut-points
\begin{table}[htbp]
\centering
\caption{Birth-weight quantile cut points and Dirichlet priors}
\label{tab:birthweight_quantiles}
% siunitx is needed only for the S columns
\begin{tabular}{@{}l c S[table-format=2.2] c S[table-format=2.2]@{}}
\toprule
& \multicolumn{2}{c}{\textbf{LBW + Normal}} &
  \multicolumn{2}{c}{\textbf{LBW only}} \\
\cmidrule(lr){2-3}\cmidrule(lr){4-5}
\textbf{Quantile} &
  \textbf{Range (g)} & {\textbf{Prior (\%)}} &
  \textbf{Range (g)} & {\textbf{Prior (\%)}} \\
\midrule
Q1     &  227--1170 & 0.84 &  227--1170 & 10 \\
Q2     & 1170--1644 & 0.84 & 1170--1644 & 10 \\
Q3     & 1644--1899 & 0.83 & 1644--1899 & 10 \\
Q4     & 1899--2069 & 0.83 & 1899--2069 & 10 \\
Q5     & 2069--2183 & 0.87 & 2069--2183 & 10 \\
Q6     & 2183--2270 & 0.83 & 2183--2270 & 10 \\
Q7     & 2270--2350 & 0.86 & 2270--2350 & 10 \\
Q8     & 2350--2410 & 0.93 & 2350--2410 & 10 \\
Q9     & 2410--2460 & 0.71 & 2410--2460 & 10 \\
Q10    & 2460--2500 & 0.80 & 2460--2500 & 10 \\
Normal & \textgreater{}2500 & 91.67 & & \\
\bottomrule
\end{tabular}
\end{table}

Table~\ref{tab:birthweight_quantiles} the two types include: LBW and NBW, and the other restricted only to LBW, where NBW is defined as any newborn with greater than 2.5 kg at birth \parencite{wiki:nbw}. Once the tree is fit, they are called the "full" and "LBW-only" models respectively. The NBW observations are aggregated into one column called \texttt{counts\_above\_2.5kg}, and will serve as the eleventh birth-weight category in the consolidated counts dataset. When given to CART, it is of primary concern how the inclusion of this column changes the tree construction, variable selection, and stability of estimates. Additionally, the prior construction is heavily skewed in the full model, where the NBW column has probability of 91.67\%, while the LBW-only is a uniform 10\% prior probability across all ten LBW categories by construction.

In summary, this preprocessing consolidation yields ten discretized quantile birth-weight categories used to allocate all observations into \emph{counts}. This provides a detailed gradations of the LBW region, and adding the aggregated NBW category provides the full range of birth-weight outcomes in the dataset. This approach prevents scarcity in any one category, and the data will be called \emph{counts data} from here forward.