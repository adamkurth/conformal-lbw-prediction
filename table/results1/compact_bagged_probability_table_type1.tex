% Compact table comparing lowest and highest bagged probability classes
% Add these packages to your LaTeX preamble:
% \usepackage{booktabs}
% \usepackage{siunitx}
% \usepackage{caption}
% \usepackage{threeparttablex}

\begin{table}[htbp]
\centering
\caption{Comparison of Common Predictors Between Lowest and Highest Bagged Probability Classes}
\label{tab:bagged_probability_comparison}
\begin{threeparttable}
\begin{tabular}{lcc}
\toprule
Predictor & Lowest 5 & Highest 5
\\
\midrule
Mean Bagged Probability & 0.0617 & 0.1799
\\
Mean 95\% CI & [0.0578, 0.0648] & [0.1532, 0.2666]
\\
Mean CI Width & 0.0070 & 0.1134
\\
\midrule
\textbf{Sex} & \textbf{Male} & \textbf{Female}
\\
\textbf{Marital Status} & \textbf{Married} & \textbf{Not Married}
\\
\textbf{Race (Black)} & \textbf{Not Black} & \textbf{Black}
\\
Age > 33 & \textemdash & \textemdash
\\
High School Ed. & \textemdash & \textemdash
\\
Full Prenatal & \textemdash & \textemdash
\\
\textbf{Smoker} & \textbf{No} & \textbf{Yes}
\\
\bottomrule
\end{tabular}
\begin{tablenotes}[flushleft]
\small
\item \textit{Note}: Bold values indicate predictors that are consistent across all classes within the group.
\item ``\textemdash'' indicates mixed values within the group.
\item CI = Confidence Interval for bagged probability estimates.
\end{tablenotes}
\end{threeparttable}
\end{table}
