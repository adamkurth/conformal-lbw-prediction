% insert table of quantile cutpoints
\begin{table}[htbp]
\centering
\caption{Birth-weight quantile cut points and Dirichlet priors}
\label{tab:birthweight_quantiles}
% siunitx is needed only for the S columns
\begin{tabular}{@{}l c S[table-format=2.2] c S[table-format=2.2]@{}}
\toprule
& \multicolumn{2}{c}{\textbf{Type 1: LBW + Normal}} &
  \multicolumn{2}{c}{\textbf{Type 2: LBW only}} \\
\cmidrule(lr){2-3}\cmidrule(lr){4-5}
\textbf{Quantile} &
  \textbf{Range (g)} & {\textbf{Prior (\%)}} &
  \textbf{Range (g)} & {\textbf{Prior (\%)}} \\
\midrule
Q1     & 227--1170 & 0.84 & 227--1170 & 10.00 \\
Q2     & 1170--1644 & 0.84 & 1170--1644 & 10.00 \\
Q3     & 1644--1899 & 0.83 & 1644--1899 & 10.00 \\
Q4     & 1899--2069 & 0.83 & 1899--2069 & 10.00 \\
Q5     & 2069--2183 & 0.87 & 2069--2183 & 10.00 \\
Q6     & 2183--2270 & 0.83 & 2183--2270 & 10.00 \\
Q7     & 2270--2350 & 0.86 & 2270--2350 & 10.00 \\
Q8     & 2350--2410 & 0.93 & 2350--2410 & 10.00 \\
Q9     & 2410--2460 & 0.71 & 2410--2460 & 10.00 \\
Q10     & 2460--2500 & 0.80 & 2460--2500 & 10.00 \\
Normal & \textgreater{}2500 & 91.67 & -- & -- \\
\bottomrule
\end{tabular}
\end{table}
